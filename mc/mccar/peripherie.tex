% coding:utf-8

%----------------------------------------

\documentclass[a4paper,10pt,fleqn]{article}

\usepackage{layout}

\title{Peripherie MC Car}

\author{Daniel Winz}
\date{\today~\dtc}

\begin{document}

\maketitle

\newpage

\tableofcontents

\newpage

\section{Übersicht}
Das MC Car beinhaltet verschiedene Peripherie. Dieses Dokument soll einen 
Überblick über die einzelnen Bestandteile der Peripherie und deren Ansteuerung 
geben. Auf dem MC Car ist folgende Peripherie vorhanden. 
\begin{itemize}
  \item LED \\
        I/O (PWM)
  \item Joystick \\
        I/O
  \item Motor mit Encoder \\
        Motor: I/O (PWM) \\
        Encoder: I$^2$C / SPI zu Encoder $\mu$C
  \item Buzzer \\
        I/O (PWM)
  \item Linien Sensoren \\
        I/O / ADC
  \item IR Kommunikation und Objekterkennung \\
        I/O / Timer
  \item Farbsensor mit LED \\
        I/O / I$^2$C
  \item Beschleunigungssensor \\
        I$^2$C
  \item OLED Display (option) \\
        I$^2$C
  \item Bluetooth Modul (option) \\
        I$^2$C
\end{itemize}

\section{LED}
Auf dem MC Car sind sowohl vorne als auch hinten LEDs vorhanden. Diese können 
wie folge mittels der Parallelen I/O Ports angesteuert werden. \\
\begin{table}[h!]
\begin{tabular}{lll}
\rowcolor{white} LED                 & Pin   & aktiv \\
\rowcolor{lgray} Rear                & PTD2  & 1     \\
\rowcolor{white} Front Left  Red     & PTF1  & 0     \\
\rowcolor{lgray} Front Left  Green   & PTC4  & 0     \\
\rowcolor{white} Front Left  Blue    & PTF0  & 0     \\
\rowcolor{lgray} Front Right Red     & PTF2  & 0     \\
\rowcolor{white} Front Right Green   & PTC6  & 0     \\
\rowcolor{lgray} Front Right Blue    & PTE7  & 0     \\
\end{tabular}
\end{table}

\section{Joystick}
Auf dem MC Car ist ein Joystick verbaut. Bei Betätigung werden die 
entsprechenden Leitungen mit Masse verbunden. Dafür müssen die internen 
Pull-up Widerstände eingeschaltet werden. 
\begin{table}[h!]
\begin{tabular}{llll}
\rowcolor{white} Richtung    & PTG0  & PTG1  & PTG2  \\
\rowcolor{lgray} Up          & 0     & 0     & 1     \\
\rowcolor{white} Right       & 1     & 1     & 0     \\
\rowcolor{lgray} Down        & 0     & 1     & 1     \\
\rowcolor{white} Left        & 1     & 0     & 1     \\
\rowcolor{lgray} Push        & 0     & 1     & 0     \\
\end{tabular}
\end{table}

\section{Motor mit Encoder}
\begin{table}[h!]
\begin{tabular}{lllll}
\rowcolor{white} Aktiver Motor      & PTD4  & PTD5  & PTD6  & PTD7  \\
\rowcolor{lgray} Rechts vorwärts    & 0     & 1     & 0     & 0     \\
\rowcolor{white} Rechts rückwärts   & 1     & 0     & 0     & 0     \\
\rowcolor{lgray} Links vorwärts     & 0     & 0     & 1     & 0     \\
\rowcolor{white} Links rückwärts    & 0     & 0     & 0     & 1     \\
\end{tabular}
\caption{Direkte Ansteuerung}
\end{table}
\begin{table}[h!]
\begin{tabular}{lllllll}
\rowcolor{white} Aktiver Motor      & PTD4  & PTD5  & PTD6  & PTD7  & PTF4  & PTF5  \\
\rowcolor{lgray} Rechts vorwärts    & 0     & Z     & 0     & 0     & PWM   & PWM   \\
\rowcolor{white} Rechts rückwärts   & Z     & 0     & 0     & 0     & PWM   & PWM   \\
\rowcolor{lgray} Links vorwärts     & 0     & 0     & Z     & 0     & PWM   & PWM   \\
\rowcolor{white} Links rückwärts    & 0     & 0     & 0     & Z     & PWM   & PWM   \\
\end{tabular}
\caption{Ansteuerung mit PWM}
\end{table}



\section{Buzzer}

\section{Linien Sensoren}

\section{IR Kommunikation und Objekterkennung}

\section{Farbsensor mit LED}

\section{Beschleunigungssensor}

\section{OLED Display}

\section{Bluetooth Modul}

\end{document}

